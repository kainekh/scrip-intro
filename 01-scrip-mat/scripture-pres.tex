\documentclass{beamer}
\usepackage[T1]{fontenc}
\usepackage[utf8]{inputenc}
\usepackage{pslatex}
\usepackage[greek.polutoniko,hebrew,english]{babel}
\usepackage{cjhebrew}
\usetheme{Berkeley}
\usecolortheme{beetle}

\title{Scripture}

\begin{document}

\maketitle

\section{Atoms}

\begin{frame}
  ``Atom'' has a primarily scientific meaning in our culture.
  It comes form early philosophers for the basic building blocks of nature.
  It may also be used as the fundamental unit of other disciplines as well.
\end{frame}

\begin{frame}
  Our fundamental atoms are:\pause
  \begin{itemize}
	\item Terminology\pause
	\item Context\pause
	  \begin{itemize}
		\item Historical\pause
		\item Liturgical\pause
	  \end{itemize}
	\item Technology
  \end{itemize}
\end{frame}

\begin{frame}
  \begin{itemize}
	\item The English word ``Scripture'' is a loanword from the Latin \emph{scriptura}.\pause
	\item \emph{Scriptura} means ``writing.''\pause
	\item It translates the Greek \textgreek{γραφή}, which also means ``writing.''
  \end{itemize}
\end{frame}

\begin{frame}
  The word is usually used in the plural in the NT.
  It did not denote a single book.
  It denoted an unspecified number of writings.\pause
  \begin{itemize}
	\item It is in the plural, meaning that they did not think of a single book.\pause
	\item It is unspecified, indicating either that everyone knew what was in it, it was variable, or it was simply not that important.
  \end{itemize}
\end{frame}

\begin{frame}
  Neither does ``canon'' denote the Bible.\pause
  \begin{itemize}
	\item \textgreek{κανών}, \emph{kanōn}, originally denoted a reed used for measuring length. \pause
	\item The term later expanded to include laws and lists of laws\pause
	\item From there it was taken into Christian churches and denoted ecclesiastical laws.\pause
	\item Canons of Scripture were some of those laws.
  \end{itemize}
\end{frame}

\begin{frame}
  Nobody told us why or how the books were canonized.
  Odds are, the criteria varied with both time and place.
  Some of the common theories include (not mutually exclusive):\pause
  \begin{itemize}
	\item Being read in the churches\pause
	\item Being prescribed by leaders\pause
	\item Being used by laity\pause
	\item Being used to establish both doctrine and other canons.
  \end{itemize}
\end{frame}

\begin{frame}
  The word ``canon'' became narrowed to the Bible only very recently.
  Using it that way divorces the Bible from its history.
\end{frame}

\begin{frame}
  The NT is largely agreed on.
  The Syrian Churches adopted the common NT relatively recently.
  Now everybody but the Ethiopians has the same NT, but it is also technically not common to all Christian groups.
\end{frame}

\begin{frame}
  The first instance of the modern NT was in St. Athanasius' Paschal letter in 317.
  Prior to this, all examples differed slightly or radically from the received NT.
  They may have some books we don't have (e.g. the Shepherd) or miss ones we have (e.g. Revelation).
\end{frame}

\begin{frame}
  The Old Testament canon has never been settled.
  The Roman Catholics settled it for themselves at Trent.
  However, multiple Old Testaments persist from antiquity.
\end{frame}

\begin{frame}
  Even within the Church, there is difference over the OT.
  Greeks have 4 Macc and Russians 3 Esdras in their Bibles but separate them from the rest of the OT.
  There have been other differences over history as well.
\end{frame}

\begin{frame}
  With these issues, it becomes tricky to talk about ``the'' Old Testament.
  It may well be that we \emph{shouldn't} try to pin down the OT to one collection of books.
  If God has never bothered to do so, we shouldn't.
\end{frame}

\section{Context}

\begin{frame}
  The context of Scripture necessarily includes the following:\pause
  \begin{itemize}
	\item Context of original composition\pause
	  \begin{itemize}
		\item Language\pause
		\item Purpose\pause
		\item Culture\pause
		\item Similar literature\pause
	  \end{itemize}
	\item Context of copying
  \end{itemize}
\end{frame}

\begin{frame}
  The ``original'' context can be framed with the following shorthand questions:
  \begin{itemize}
	\item \textbf{Who} wrote it?\pause
	\item \textbf{When} was it written?\pause
	\item \textbf{Where} was it written?\pause
	\item \textbf{To whom} was it written?\pause
	\item \textbf{Why} was it written?\pause
	\item \textbf{What genre} was it written in?\pause
	\item \textbf{What language} was it written in?
  \end{itemize}
\end{frame}

\begin{frame}
  The works were written to communicate ideas.
  This sense is indespensible.
  If your landlord wrote you a ``Where's my money?'' letter, whatever else you make of it, if you don't read it as intended you're on the street.
\end{frame}

\begin{frame}
  We must try to discern the context as best we can.
  To that end, we must use good sources: reputable commentaries, journals, and so on.
  Discerning the context is a scientific enterprise, open to change.
\end{frame}

\begin{frame}
  For \emph{this} level of understanding Scripture, the Fathers are less useful.
  We have knowledge of original related languages, ancient (even for them) legal documents, related authors, sources for the biblical authors, and so on.
  They did not have these.
  They are, however, of great use in other methods and levels of understanding Scripture.
\end{frame}

\begin{frame}
  The questions are obvious questions, but they lead to other questions.
  For example:
  \begin{itemize}
	\item Is the phrase \cjRL{'elEh twol:dwot}, \emph{ēle tōl\scriptsize e\normalsize dōt}, a section division?\pause
	\item Are there legal formulas anywhere, such as when Sarah offers her handmaid?\pause
	\item Does Jacob's ladder have any parallels in conventions of the period?
  \end{itemize}
\end{frame}

\begin{frame}
  However, context doesn't end at its composition.\pause
  \begin{itemize}
	\item Each step in the editing adds a context.\pause
	\item Its transmission adds to the context.\pause
	\item The liturgical context that justified it adds to the context.\pause
	\item Collecting it into a set of texts adds to context.
  \end{itemize}
\end{frame}

\begin{frame}
  Technology is also context.
  Think of music.\pause
  \begin{itemize}
	\item Vinyl\pause
	\item Cassettes\pause
	\item CDs\pause
	\item Digital
  \end{itemize}
\end{frame}

\begin{frame}
  Scrolls were like cassettes.
  You cannot skip around.
  You must stop, rewind, and fast-forward to get to locations.
  This facilitates making allusions over quotations as well as remembering themes and structures.
  You cannot turn to Numbers 16, so you had better keep in mind enough of what happened with Korah.
\end{frame}

\begin{frame}
  Codices are like CDs.
  The codex allowed people to skip around and cross-reference with ease.
  The TOCs were rough but gave locations as well.
  It was not able to give fine-tuned cross-references.
\end{frame}

\begin{frame}
  The printing press allowed for very fine-tuned page setting.
  We could generate indices.
  The modern book made incremental improvements on the codex while vastly improving efficiency.
\end{frame}

\begin{frame}
  Digital resources allows us to count every time a word appears.
  They will instantly look up a word in a dictionary or related documents.
  We can run several editions of a text in parallel that sync and scroll together.
  We can have pages of notes and still present mainly the biblical text because notes can be hidden (e.g. NET Bible).
\end{frame}

\begin{frame}
  We need to keep this in mind when we read.
  If we use systematic word studies across dozens of documents, then we are \emph{not} reading it like someone in the first century.
  In fact, we must do such things and \emph{cannot} read it the way they read things.
\end{frame}

\begin{frame}
  Religious texts were written with liturgical contexts in mind and/or were preserved in those contexts.
  This would be inevitable because:\pause
  \begin{itemize}
	\item They read the texts liturgically.\pause
	\item Modern freeform worship had not been invented.\pause
	\item Liturgy was seen as manifesting the reality of the divine council on earth.
  \end{itemize}
\end{frame}

\begin{frame}
  Authorship operated differently:\pause
  \begin{itemize}
	\item Authors would regularly keep themselves anonymous, being only identified by tradition.\pause
	\item Copyists had the right to change it to a limited (and unknown how limited) extent.\pause
	\item Subsequent editions could be redistributed under the author's aegis (c.f. our practice of ``editions'').
  \end{itemize}
\end{frame}

\begin{frame}
  Ancients would probably have seen our emphasis on individual authorship as both bizarre and insane.
  They may have even considered it immoral.
  These differences have produced different sets of abuses and strengths.
\end{frame}

\begin{frame}
  Our view tends to atomize creativity.
  We lock it away and break the train of transmission.
  For example Disney princesses are rooted in and based on folk tales.
  However, imagine if you created a derivative of their work.
  As a result, the sphere of genuine creativity retracts, and we shame people for doing things that would previously considered moral.
  Today Matthew and Luke would be considered immoral and, quite possibly, \emph{illegal}.
\end{frame}

\begin{frame}
  It has enshrined the vice of acquisitiveness, \textgreek{πλεονεξία}.
  When we publish something, we are acquisitive and greedy.
  ``It's \emph{my} work.
  \emph{You} can't share it.
  \emph{You} can't change it.''
  This sort of attitude is alien to ancient creation, and it is evil.
\end{frame}

\begin{frame}
  On the other hand, we have strong protections against malevolent pseudepigrapha and slander.
  No better example of how successful it could be in antiquity exists than Origen of Alexandria.
  Origen was slandered for teaching several things, and people began following them (see the Origenist articles of II Constantinople for an example of how far it had gotten).
  So not only could a man be lied about, but his name could be so inextricably linked to a lie it could not be rehabilitated (lifting the anathema against Origen would justify the abuses 2C tried to restrain).
\end{frame}

\begin{frame}
  Ancient views had their abuses as well.
  People could write in another's name with impunity both with good and evil motives.\pause
  \begin{itemize}
	\item They could hurt someone by changing the work.\pause
	\item They could write in that person's name for both good and evil.\pause
	\item Such imputations could be permanent.\pause
	  \begin{itemize}
		\item Anaximenes\pause
		\item Origen\pause
		\item Paul 2 Thess 2.2
	  \end{itemize}
  \end{itemize}
\end{frame}

\begin{frame}
  Practical representitives:\pause
  \begin{itemize}
	\item Chronicles, 1 Esdras, Ezra-Nehemiah\pause
	\item Matthew, Mark, Luke\pause
	\item Daniel
  \end{itemize}
\end{frame}

\begin{frame}
  Genesis, and the rest of the Torah, bear this mark as well.
  There are signs of placename updates, anachronistic names, variations in vocabulary and style.
\end{frame}


\section{Text Traditions}

\subsection{Old Testament}

\begin{frame}
  Variations creep in either by errors or by deliberate variation (those accepted and rejected).
  We do not have the originals.
  I call the lines ``traditions'' because they are not simply copies.
  They represent visions of the form of the text.
  However, we may affirm them all; there need not be a single vision, but we do have to rely primarily on one as a reference.
  That tradition will form us.
\end{frame}

\begin{frame}
  The Scripture is preserved for us in several major versions.
  For the Old Testament our most important versions:\pause
  \begin{itemize}
	\item Masoretic text\pause
	\item Samaritan Pentateuch\pause
	\item Septuagint\pause
	\item Targums
  \end{itemize}
\end{frame}

\subsubsection{Septuagint}

\begin{frame}
  The Septuagint is the most important for our interests.
  The New Testament quotes it roughly 70\% of the time where it varies with the Masoretic Text.
  These quotations are not evenly distributed in the LXX.
  It is also the version used to settle the christological debates of the first millennium that forged the faith.
  It is abbreviated as LXX.
\end{frame}

\begin{frame}
  The LXX was originally just the Torah.
  It was translated around Alexandria, Egypt in the third century BC.
  Our main account is the \emph{Letter of Aristeas}.
  The account was made more fantastic as time went on.
\end{frame}

\begin{frame}
  As time went on other books were assembled and translated and transmitted with the original LXX.
  These are, technically, the ``Old Greek'' (excepting Daniel), and ``Septuagint'' denoted the Torah proper.
  Over time, the whole collection came to be called by the name ``Septuagint.''
  I will usually use it in the messy and ambiguous way to denote the whole Old Testament.
\end{frame}

\begin{frame}
  This does not mean there was a single collection passing as the LXX.
  There were varied and contradictory lists.
\end{frame}

\begin{frame}
  The LXX texts do not all represent a single translation.\pause
  \begin{itemize}
	\item The books of 1-4 Kingdoms (1-2 Sam and 1-2 Kings in the MT) represents two different translations mashed together in a seemingly haphazard way.\pause
	\item Daniel comes to us in two translation.
	  The Old Greek translation was substantially different than a later translation called ``Theodotion.''
	  Later the Church replaced the OG with T.
	  The NT depends on the OG when it uses Daniel in the LXX.\pause
	\item By far, the most significant change is the Hexapla.
  \end{itemize}
\end{frame}

\begin{frame}
  The Hexapla was a recension created by Origen of Alexandria.
  It was a six-column attempt at a critical version.
  He included the LXX, a proto-MT, and several others.
  In the final column he would put them together, tending to combine rather than take away.
  His marks showed what he changed.
  The marks were lost in later mss., and his combined version made its way into the mainstream.
  Very little of the Hexapla remains, unfortunately.
\end{frame}

\begin{frame}
  The readings of the LXX go back to a Hebrew original.
  For instance, in Qumran there are three Jeremiah mss.
  When the LXX differs with the MT, it agrees with the SP more often than not.
  Since we cannot always know if it was changed in translation, unless I have reason to suspect otherwise, I assume it represents a different Hebrew \emph{vorlage}.
\end{frame}

\subsubsection{Masoretic Text}

\begin{frame}
  The Masoretic text is the standard Bible of Judaism.
  With some modifications, it forms the Old Testament of Protestantism.
\end{frame}

\begin{frame}
  Though people in my Church often say that it is a late text, readings that agree with the MT comprise a large portion of the Dead Sea Scrolls.
  We thus have a proto-Masoretic text in the DSS.
  It is not identical, but it is close enough to count for identity.
\end{frame}

\begin{frame}
  Over time, the text became more standardized at a school, not a council, in Jamnia.
  They began adding vowel markings.
  Vowel markings began centuries earlier with making some letters double as vowels.
  Here they started using various systems of dots.
  The ``Tiberian'' system has become dominant in printed Bibles.
  The standardization process finished about a thousand years ago.
\end{frame}

\begin{frame}
  This process also finalized the Jewish list of books.
  Previously some books were in the air, such as Ecclesiastes and Esther.
  There were other books that previously had been venerated and had an impact, such as Sirach.
\end{frame}

\begin{frame}
  In the Reformation, the Protestants came to believe that since the Jews didn't have the ``apocrypha'' they were suspect or uninspired.
  The logical flaw is that the Christian canon and Hebrew Bible were both set in place after Christianity and Judaism parted ways.
  The books were originally set aside between the testaments following St. Jerome's practice.
  The final straw was that they were omitted to reduce printing costs.
  Now we have arrived at a point where people consider them wholly uninspired.
\end{frame}

\begin{frame}
  The changes to the Hebrew Bible made in this process were fairly limited.
  Ezra-Nehemiah were split and made two different books.
  This was taken from the Vulgate and that from Origen who believed they should be separate.
  They are still joined in Hebrew Bibles and the Septuagint.
  They are divided into 1 and 2 Esdras (Vulgate numbering; I use the LXX where they are together 2 Esdras).
  Most English Bibles, Christian and Jewish, follow Origen's division and make this change to the text.
\end{frame}

\begin{frame}
  Book names are taken from the LXX.
  We call the names the Torah Genesis, Exodus, and so on instead of Hebrew names like Bereshith and Hashamoth.
\end{frame}

\begin{frame}
  In key passages, translations use the LXX to supercede and replace the Hebrew.
  In Isaiah 7.14, conservative English Bibles say ``virgin.''
  The Hebrew is ``young woman.''
  In Habakuk 2.4 the righteous man is saved by his faithfulness.
  English Bibles use the LXX which can mean ``faithfulness'' but can also mean ``faith'' to opt for the latter.
\end{frame}

\begin{frame}
  Thus even English translations that aim for the Hebrew still are mixtures where the LXX supercedes the Hebrew in several places.
  If a translation strictly from the Hebrew is desired, nothing beats a strictly Jewish translation.
\end{frame}

\subsection{Samaritan Pentateuch}

\begin{frame}
  The Samaritan Pentateuch is the Samaritans' version of the Law.
  It has primitive readings, but it is most famous for changing the place that the \textsc{Lord} would choose was Gerazim.
  Other minor differences appear.
  For instance, they repoint any passage God appears so that it is not truly God appearing.
\end{frame}

\subsubsection{Dead Sea Scrolls}

\begin{frame}
  The Dead Sea Scrolls are a collection of manuscripts of both the Bible and those of a local sect that were found in the mid-twentieth century.
  The Biblical manuscripts represent copies of the Bible and related texts that date to the first couple of centuries BC.
\end{frame}

\begin{frame}
  These manuscripts contain a textline that agrees with the Masoretic.
  The most famous of this line is the Great Isaiah Scroll.
  It contains texts that agree with the Septuagint and Samaritan Pentateuch as well.
  Some texts are unaligned.
\end{frame}

\begin{frame}
  They seem to have been indifferent on the matter.
  One can find multiple recensions of a text side-by-side.
  The question, ``Which one is the original'' didn't bother them, and that should be a sign for us today.
\end{frame}

\subsubsection{Targums}

\begin{frame}
  The Targums are translations of the Hebrew Bible into Palestinian Aramaic.
  The translations tend more toward paraphrase in most cases.
  However, they give vent to ideas that were current.
\end{frame}

\begin{frame}
  They were also composed after our period of the NT.
  However, there is no reason to suppose that they contain only later ideas.
  Indeed, some of the ideas demonstrate tremendous continuity with NT thought.
\end{frame}

\subsubsection{Others}

\begin{frame}
  There are other texts of the OT that are not covered here:\pause
  \begin{itemize}
	\item The Peshitta\pause
	\item The Vulgate\pause
	\item the Coptic
  \end{itemize}
\end{frame}

\begin{frame}
  These are, for the most part, of less interest.
  Most are translations of the Septuagint.
  The Vulgate is a direct witness to the Hebrew text of St. Jerome's day, but it was mostly standardized by that point.
\end{frame}

\subsection{New Testament}

\begin{frame}
  The New Testament is, textually, in far better shape.
  Old Testament revisions occur further back than we can see.
  NT revisions occur more recently.
  In most cases, we can successfully reconstruct fairly close to an urtext.
\end{frame}

\begin{frame}
  It becomes a relevant question in the NT:\\
  Do we use the ``original'' text if we can reconstruct it?
\end{frame}

\begin{frame}
  The Greek text comes down to us in several basic textlines.\pause
  \begin{itemize}
	\item Byzantine\pause
	\item Alexandrian\pause
	\item Western\pause
	\item Caesarean
  \end{itemize}
\end{frame}

\begin{frame}
  The most important for our purposes are the Byzantine and Alexandrian.
  Modern eclectic texts rely mostly on the Alexandrian.
  However they aren't limited to it.
\end{frame}

\begin{frame}
  The Byzantine is a later text.
  It is the Christian equivalent of the Masoretic text.
  It is stable with few changes once it's established (Revelation is the biggest exception).
  That text form became the stable text used in liturgies and theology after about the fifth century.
\end{frame}

\begin{frame}
  Modern scholarly texts are eclectic texts that attempt to reconstruct the earliest possible reading for the NT.
  This is not done with the OT, but there is a project that is doing so.
  There specific rules that are used by textual scholars to evaluate readings.
  It is not black and white, but they have enough predictive power that several readings have been anticipated then found in papyri.
  They are reliable reconstructions of the early text.
\end{frame}

\begin{frame}
  For my part, I use a Byzantine text for anything liturgical or theological and a critical text for anything scholarly.
  The duality is a respect for the structure that God has chosen to preserve things for us.
  However, I do not feel that I can ignore scholarly advancements and discoveries.
\end{frame}

\section{conclusion}

\begin{frame}
  This covers the physical history of the Bible very briefly.
  I did not mention much on its inspiration or ranges of interpretation.
  I'd never keep it in a remotely reasonable time frame if I tried that too.
  That is next, then we get to Genesis.
\end{frame}

\end{document}
