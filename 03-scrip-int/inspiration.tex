\documentclass{beamer}
\usepackage[T1]{fontenc}
\usepackage[utf8]{inputenc}
\usepackage{pslatex}
\usepackage[greek.polutoniko,hebrew,english]{babel}
\usepackage{cjhebrew}
\usetheme{Berkeley}
\usecolortheme{albatross}

\title{Interpretation of the Bible}
\author{Kenneth Gardner}

\begin{document}

\maketitle

\begin{frame}
  Biblical interpretation involves:\pause
  \begin{itemize}
	\item Extracting the meanings of the biblical text\pause
	\item Applying the meanings to our context.
  \end{itemize}
\end{frame}

\begin{frame}
  If the meaning is not properly extracted, then the application is likely to be wrong.
\end{frame}

\begin{frame}
  If there is no application, then interpretation is just trivia.
\end{frame}

\begin{frame}
  I lean heavily on the first.\pause
  \begin{itemize}
	\item It takes a lot of time to do the first.\pause
	\item Listeners are anonymous, and so I can't cater it to their concerns easily.
  \end{itemize}
\end{frame}

\begin{frame}
  How we approach Scripture should reflect what we believe its nature is.
\end{frame}

\begin{frame}
  If Scripture is a product of both human and divine activity, then interpretation needs to focus on both human and divine aspects.
\end{frame}

\begin{frame}
  It is possible to look at Scripture as if it had no temporal or culturally-dependent issues.
  This approach is fundamentally gnostic and has no history.
\end{frame}

\begin{frame}
  We can limit the meaning to the culture as if God could not move something beyond it.
  This approach generally imports the assumptions of positivism.
\end{frame}

\begin{frame}
  To balance this, there have been historically three broad approaches to the text:\pause
  \begin{itemize}
	\item Literary, or ``literal''\pause
	\item Symbolic or allegorical\pause
	\item Ethical
  \end{itemize}
\end{frame}

\begin{frame}
  Some models subdivide much more than this.
  However I prefer the KISS principle.
\end{frame}

\begin{frame}
  I covered the literary approach in my material history video.
  This approach looks at the language, idioms of the day, technology of the day, cultural conventions, literary norms, genre, probable intentions of the author, original purpose, et al.
\end{frame}

\begin{frame}
  The allegorical approach assumes that there is symbolism intended by God that the original audience did not know but becomes evident through a later event, generally the life of Christ.
\end{frame}

\begin{frame}
  The ethical approach uses the first two in order to extract answer to questions about ``What ought I to do?''
\end{frame}

\begin{frame}
  Each one is fraught with dangers if we emphasize it too much, and people \emph{have} emphasized them too much in history with disastrous consequences.
\end{frame}

\begin{frame}
  If we emphasize the ``literal'' meaning too much, then we undermine the authority of the apostles and ultimately the Scripture itself.
\end{frame}

\begin{frame}
  If we emphasize allegory too much, then the Scripture becomes a megaphone for whatever we want to read into it.
\end{frame}

\begin{frame}
  If we emphasize the ethical too much, then the Scripture is reduced to a magic 8-ball.
\end{frame}

\section{literal}

\begin{frame}
  By ``literary'' interpretation, I do not mean ''literal.''\pause
  \begin{itemize}
	\item Literary emphasizes genre over plain meaning.\pause
	\item It recognizes that much of what people say is said in symbolic language.\pause
	\item It recognizes that what should be plainly literal will change from period and culture to period and culture.\pause
	\item It focuses on what the author intends and is not reactionary.
  \end{itemize}
\end{frame}

\begin{frame}
  Literalism is a largely modern approach.\pause
  \begin{itemize}
	\item It originated as an attempt to shore up Scripture against 19th century liberalism.\pause
	\item It appropriates the values of the Enlightenment and nominalism and makes them binding for Scripture.
  \end{itemize}
\end{frame}

\begin{frame}
  As a result, literalism is\pause
  \begin{itemize}
	\item Reactionary and always changing.\pause
	\item Brittle\pause
	\item Inherently modernist\pause
	\item Reductionist\pause
	  \begin{itemize}
		\item It confuses the elements with the intent or authority of the passage.\pause
		\item It reduces and eliminates any meaning that is not surface-apparent.
	  \end{itemize}
  \end{itemize}
\end{frame}

\begin{frame}
  Literary interpretation is not timeless but progressive.\pause
  \begin{itemize}
	\item It relies on what we know about the past; more data is continually being unearthed.\pause
	\item It relies on analysis of the language, and languages change so that those are constantly being hidden and uncovered.\pause
	\item It depends on specialists.
  \end{itemize}
\end{frame}

\begin{frame}
  Literary meanings are not beshadowed so that they are available to everybody who is willing to study.
\end{frame}

\begin{frame}
  The result is that newer resources are better than older ones for \emph{this} approach to Scripture.
\end{frame}

\begin{frame}
  It requires hard work and resources.
\end{frame}

\section{Allegorical}

\begin{frame}
  The allegorical interpretation is the most strange to us today.
  We are all of us biased toward materialism and nominalism.
\end{frame}

\begin{frame}
  It is the most widely neglected, even disliked interpretation system.
\end{frame}

\begin{frame}
  It is also necessary for Christianity to be true.
\end{frame}

\begin{frame}
  The allegorical approach assumes that there is a meaning in the text that can only be seen if we look with the appropriate lens.
\end{frame}

\begin{frame}
  This is the sense of Scripture that is not available or most nonsensical to unbelievers.
\end{frame}

\begin{frame}
  The lenses include:\pause
  \begin{itemize}
	\item Typology\pause
	\item Allegory\pause
	\item Bible-wide themes or word-studies
  \end{itemize}
\end{frame}

\begin{frame}
  Typology and allegory are fundamentally the same but with more detail in the latter.
  Having one gives the other.
\end{frame}

\begin{frame}
  They were later systematized so that the terms came to be technically distinct (and other forms of interpretation beside them).
  I'm keeping it simple.
\end{frame}

\begin{frame}
  The word ``type'' became a literary and philosophical term through Plato.
  All objects we see are a ``type'' or ``shadow'' of some fuller \textgreek{a>i'wnion} reality.
\end{frame}

\begin{frame}
  When we see a chair, we see a type of chair.
  We recognize it as ``chair'' because our \textgreek{no~us} recognizes the \textgreek{a>i'wnion} archetype of chair.
\end{frame}

\begin{frame}
  The archetypical chair doesn't exist in time and so doesn't change or progress.
  The timelessness is called an \textgreek{a>'iwn}.
  We translate that as ``eternal.''
\end{frame}

\begin{frame}
  A related term is ``antitype.''
  It originally meant ``opposite'' but could mean something foreshadowed by a type.
\end{frame}

\begin{frame}
  The
  \selectlanguage{greek}
  no~us
  \selectlanguage{english}
  is the mind in the sense that it is the function of our intellect that perceives and receives truth.
  It is \emph{not} ``mind'' in the sense of calculating or deliberating.
  This is \textgreek{diano~ia}, ``through-mind.''
\end{frame}

\begin{frame}
  This sense of ``mind'' is what St. Paul intends when he says, ``Be transformed by the renewing of your mind'' in Romans 12.2.
\end{frame}

\begin{frame}
  He doesn't mean learning facts or arguments.
  He means changing the very way we perceive the world.
\end{frame}

\begin{frame}
  He means how we perceive the world.
  Perception is prerational and not easily changed by propositions and arguments.
  Perception of types is part of that.
\end{frame}

\begin{frame}
  Typology also appears explicitly in Scripture.
\end{frame}

\begin{frame}
  In 1 Peter 3.21 baptism is the antitype for Flood, meaning the Flood is a type for baptism.
\end{frame}

\begin{frame}
  The Anointed also died once for sins, the just for the unjust, so that he might precede you all to God.
  Being killed in the flesh then being made alive in spirit;
  in which also he also went and proclaimed to spirits in prison,
  that were disobedient when the patience of God was waiting in the days Noah made the ark in which a few, that is eight lives were saved by water;
  which is an antitype of baptism which now also saves us not by putting away dirt but a pledge of a clear conscience for God through the resurrection of Jesus the anointed.
\end{frame}

\begin{frame}
  Adam is a ``type'' of Christ in Romans 5.14:
  \begin{quote}
	Death reigned from Adam until Moses even on those who didn't sin like the transgression of Adam, who is a type of the one to come.
  \end{quote}
\end{frame}

\begin{frame}
  The same idea shows up with the two Adams theology of I Cor. 15.
\end{frame}

\begin{frame}
  It is not just the presence of the word.
  It appears in other places that are not typological (e.g. I Ti 4.12).
  It is \emph{how} it is used.
  The concept appears without the terminology elsewhere.
\end{frame}

\begin{frame}
  In Luke 11.29ff. Jesus says his resurrection is the ``sign of Jonah.''
\end{frame}

\begin{frame}
  Hebrews treats the Law as a shadow of Christ (Hebrews 10.1)
\end{frame}

\begin{frame}
  Paul also uses it in Colossians 2.16-17, that the feast days and food laws were shadows of Christ.
\end{frame}

\begin{frame}
  Allegory is referred to explicitly once in the Bible.
  St. Paul uses it as a way to describe Sarah and Hagar in Gal 4.24.
  \begin{quote}
	Which are an allegory, for these are two covenants\ldots ''
  \end{quote}
\end{frame}

\begin{frame}
  St. Paul argues that they represent two mountains and two covenants.
\end{frame}

\begin{frame}
  Here we get to two imbalances in approaching the Bible.
\end{frame}

\begin{frame}
  One is to use allegory to undermine the literary reading of the Bible.
  A good example of this is the Epistle of Barnabas.
  Barnabas denied that Moses gave any command about food (Ep.Barn 10).
\end{frame}

\begin{frame}
  The push against the scandal of particularity in the OT in some cases involved denying the OT.
  Others, like Barnabas denied the literary meanings.
\end{frame}

\begin{frame}
  Very few people today give credence to this extreme.
  It's not very important now.
\end{frame}

\begin{frame}
  The most common extreme is its polar opposite that the only plausible meaning is the original meaning.
\end{frame}

\begin{frame}
  There are several serious problems with this view:\pause
  \begin{itemize}
	\item It undermines the authority of the apostles.\pause
	\item We cannot find the original context with certainty.\pause
	\item We also find books that have been edited several times so that we have to construct a context.\pause
	\item It is as modernist and anachronistic as everything that it seeks to defend against.
  \end{itemize}
\end{frame}

\begin{frame}
  Paul admonishes people to interpret outside the original context:\pause
  \begin{itemize}
	\item Galatians 4.24\pause
	\item I Cor. 10.11\pause
	\item II Cor 3
  \end{itemize}
\end{frame}

\begin{frame}
  There are also multiple instances where Paul admonishes people to take him or another as an example (e.g. I Cor 11.1)
\end{frame}

\begin{frame}
  This sort of interpretation were normal in the Hellenistic and Second Temple period.\pause
  \begin{itemize}
	\item The DSS include several peshers.
	  These interpret biblical books for current concerns, not for original context.\pause
	\item Greek commentators published scholia, which normalized the allegorical interpretation.\pause
  \end{itemize}
  Allegory, pesher, typology, and the like were \emph{normal} in the ancient world.
\end{frame}

\begin{frame}
  What this means is that:\pause
  \begin{itemize}
	\item Paul used a normal method of interpretation.\pause
	\item Paul admonished people to follow him.\pause
	\item Paul in several cases seems to encourage the people to use it.\pause
	\item Paul never even cautions people about his example.
  \end{itemize}
\end{frame}

\begin{frame}
  The implication of this is that, at this point, the Gospel itself \emph{depends} on the validity of allegory.
\end{frame}

\begin{frame}
  Denying allegory is a purely extrabiblical assertion.
\end{frame}

\begin{frame}
  The denial also introduces ``epicycles'' into biblical interpretation.
  It's wrong and eisegesis everywhere but where the NT books seem to do it, and they do it ubiquitously.
  It's a small step to say that using a wrong interpretation to explain the Scripture is simply wrong.
\end{frame}

\begin{frame}
  Even after denying the authority of using allegory to defend the historical context, proponents still do it.
  The most popular is to study how a word is used throughout the Bible or a biblical theme from beginning to end.
\end{frame}

\begin{frame}
  Earlier books were not written with later books in mind, nor is there any reason other than the faith to assume that such themes should be reliable.
\end{frame}

\begin{frame}
  This same assumption, that God provides this, underlies allegory.
\end{frame}

\begin{frame}
  Another reason it has been attacked is the abuses I mentioned above.
  However, the wide-spread rejection of the method means that its extremes are not a serious threat anymore.
  The opposite are.
\end{frame}

\begin{frame}
  There are several allegorical interpretations that are not in the NT that are compelling for a Christian.
\end{frame}

\begin{frame}
  Joseph is a type of Christ.\pause
  \begin{itemize}
	\item He is the favored son of his father.\pause
	\item He is betrayed by his brethren and sold to a tyrant.\pause
	\item Joseph is wrongly accused.\pause
	\item He rises from the lowest point to become second to the king (Pharaoh).\pause
	\item He delivers his people.
  \end{itemize}
\end{frame}

\begin{frame}
  Another is Joshua's battle with the Amalakites in Num 24.\pause
  \begin{itemize}
	\item Joshua's name in Greek is Jesus\pause
	\item When Moses held out his arms, Jesus prevailed.\pause
	\item Thus Jesus conquers the enemy of Moses by the sign of the cross.
  \end{itemize}
\end{frame}

\begin{frame}
  However, since this is rooted in the wording and elements outside the text, the text is not capable of checking it.
  How do we check the abuse of the method?
\end{frame}

\begin{frame}
  We must do so by watching where we get those premises from.
\end{frame}

\begin{frame}
  First, and most important, we must be christocentric.\pause
  \begin{itemize}
	\item Christ is the subject of the Scripture (Jn 5.39)\pause
	\item Christ is what makes it relevant\pause
	\item It will pass, but Christ will not.
  \end{itemize}
\end{frame}

\begin{frame}
  Secondly, it assumes Scripture is all apocalyptic.\pause
\end{frame}

\begin{frame}
  Apocalyptic reveals what happens in the heaven by means of earthly elements or symbolic elements.
\end{frame}

\begin{frame}
  Apocalyptic in its turn is rooted in liturgical worship.
  Liturgical worship gives a valid framework to interpret it in.
\end{frame}

\begin{frame}
  Historical interpretation has worked over time, and if we learn and read it, it also supplies a framework.
  We should be cautious if our allegorical arguments are radically different than what has already come.
\end{frame}

\begin{frame}
  Allegory is rooted in preconceived doctrines and beliefs, and as such should not be very progressive.
  It should develop very little.
\end{frame}

\section{Ethical}

\begin{frame}
  The ethical interpretation looks at the Bible to find out how we ought to live.
\end{frame}

\begin{frame}
  It is the easiest approach to grasp both its strengths and weaknesses.
\end{frame}

\begin{frame}
  A good example of this would be when Jesus was confronted by the Pharisees in Mk. 2.23-28.
\end{frame}

\begin{frame}
  The danger with this interpretation is illustrated by the quotations:
  \begin{itemize}
	\item Judas hanged himself.
	\item You go do likewise.
	\item What you do, do quickly.
  \end{itemize}
\end{frame}

\begin{frame}
  Another is that Ezra had the people put away foreign wives.
  If we have non-Christian wives, does it mean we must divorce them?
\end{frame}

\begin{frame}
  Throughout the OT, biblical heroes smashed idols.
  In some cases, like in Elijah, they hacked the priests to pieces.
  Should we go to a pagan temple and hack them to pieces and shatter the idols?
\end{frame}

\begin{frame}
  In each case, we would answer no. The cases are extreme, but they illustrate the problem:
  The method can lead to bad conclusions.
\end{frame}

\begin{frame}
  The answer is like that of allegory.
  First, we must emulate Christ.
  This entails non-violence and compassion.
\end{frame}

\begin{frame}
  We also have doctrines we presuppose as Christians.
  One of these is that divorce is sin, so we shouldn't put away wives.
\end{frame}

\begin{frame}
  In the place of Liturgy, we have the worshiping community.
\end{frame}

\begin{frame}
  Lastly, we have reason.
  There are actions we may want to emulate but can think through and know that it's just a bad idea.
\end{frame}

\begin{frame}
  We understand this method instinctively, and we understand the checks just as easily.
\end{frame}

\section{Conclusion}

\begin{frame}
  Most of the time, I use a literary interpretation.
  I do not avoid the other two, though.
\end{frame}

\begin{frame}
  I also make reference to history, Christian worship, and the like.
\end{frame}

\begin{frame}
  When reading, I presuppose my Church's theological context.
\end{frame}

\end{document}
