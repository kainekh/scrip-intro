\documentclass{beamer}
\usepackage[T1]{fontenc}
\usepackage[utf8]{inputenc}
\usepackage{pslatex}
\usepackage[greek.polutoniko,hebrew,english]{babel}
\usepackage{cjhebrew}
\usetheme{Berkeley}
\usecolortheme{albatross}

\title{Inspiration of Scripture}
\author{Kenneth Gardner}

\begin{document}

\maketitle

\section{Inspiration}

\begin{frame}
  \huge{Inspiration}
\end{frame}

\begin{frame}
  Before we discuss interpretation we must consider inspiration.
  The Bible, being the Book of the Church, is inspired and has a unique function.
  We must consider this function to consider properly how to approach it.
\end{frame}

\subsection{2 Peter 1.21}

\begin{frame}
  No prophecy of scripture is of any private interpretation, for prophecy was not ever borne by the will of a human but God's holy humans spoke, being born by the holy spirit.---2 Pet. 1.20-21
\end{frame}

\begin{frame}
  There are several ambiguities with this passage.\pause
  \begin{itemize}
	\item ``is not of any private interpretation'', \textgreek{>id'ias >epil'usews o>u g'inetai}\pause
	  \begin{itemize}
		\item ``is'' can mean ``became'' or even ``came''\pause
		\item ``of'' is supplied by case. It may mean ``from'' as well\pause
	  \end{itemize}
	\item What does ``borne'' denote?\pause
	\item ``holy spirit'' is indefinite\pause
	  \begin{itemize}
		\item It could be definite without the article just like ``God'' in the ``God's holy humans.''\pause
		\item It could be indefinite and refer to a reborn human spirit that is holy.
	  \end{itemize}
  \end{itemize}
\end{frame}

\begin{frame}
  I translate the ``of'' in the first clause, because it is ambiguous.
  It may mean ``of'' in the sense of origin, like ``from.''
  Meaning that it did not arise from the prophets' own interpretations.
  It may mean ``of'' in the sense that we do not have free reign in interpretation.
\end{frame}

\begin{frame}
  I take the ambiguity as deliberate.\pause
  \begin{itemize}
	\item In the immediate context, in 1.21, we see a definite reference to origin in that it is not ``by the will of a human.''\pause
	\item In 3.14-16 St. Peter also references people twisting St. Paul's words, because they have not been properly educated, meaning the other concern is enough he closes with it.
  \end{itemize}
\end{frame}

\begin{frame}
  The word ``is,'' carries the connotations of coming to be.
  The Scripture is a creature and not eternal.
  It is not the ``Word of God'' of John 1.
  It does not have the attributes of God save by participation, just like any other creature.
  This creaturely point is important to return to.
\end{frame}

\begin{frame}
  ``Borne'' appears elsewhere.
  Prophets are regularly ``borne'' or ``carried.''
  It is an apocalyptic term that implies that it is something that happens \emph{to} the prophet.\pause
  \begin{itemize}
	\item The Apostle John is in the Spirit on the Lord's day (Rev 1.10).\pause
	\item St. Paul is taken up to the third heaven. (2 Cor. 12).\pause
	\item The Prophet Isaiah is taken into the heavenly sanctuary (Is. 6).
  \end{itemize}
\end{frame}

\begin{frame}
  Most inspired authors of Scripture have had a direct divine experience.
  This experience gives them their message and authority.
  This includes when the Word of God came to someone; it was a person, not a message.
  The implication of this is that the Scripture is \emph{not} the revelation of God; that happened to the prophet.
  The Scripture is a record of the revelation of God, a revelation of the revelation.
\end{frame}

\begin{frame}
  This type of divine origin and the double entendre leads to the next potential double meaning.
  Holy people spoke as they were born by ``holy spirit.''
  This may also have a double-meaning.
\end{frame}

\begin{frame}
  It may mean directly the Holy Spirit in that it is God who reveals these things.\pause
  \begin{itemize}
	\item They do not come from the will of men.\pause
	\item The ``private interpretation'' also works well with this; they were not interpreting some divine sign on their own.
  \end{itemize}
\end{frame}

\begin{frame}
  It could refer to a reborn spirit within the prophet.\pause
  \begin{itemize}
	\item Isaiah, for instance, has the coal touch his lips to cleanse him.\pause
	\item The men on the way to Emmaus do not recognize until the breaking of the bread.\pause
	\item In this sense, it is of ``no private interpretation'' because one must be properly educated and thus in communion with the Church \emph{and} have an inner holy spirit to see things properly.
  \end{itemize}
\end{frame}

\begin{frame}
  These things indicate that proper interpretation occurs when one is properly instructed and is being taught by the Spirit or has a reborn spirit within.\\ \pause
  It is not available simply by study or by prayer but by both.
\end{frame}

\begin{frame}
  The other key passage people turn to to define the meaning of Scripture is 2 Timothy 3.16.
  However, 3.16 is not the full statement and so not sufficient to interpret it.
  \begin{quote}
	You, however, remain in what you were taught and trusted in, knowing from whom you were taught,
	and because, since infancy, you knew the holy letters, which are capable to make you wise for salvation through trusting that is in the Anointed Jesus.
	Every Scripture is inspired of God and profitable for teaching, for refutation, for correction, for educating children in righteousness,
	that the person of God may be fully qualified and equipped for every good work.
  \end{quote}
\end{frame}

\begin{frame}
  There is an order to the things he says leading up to the Scripture:\pause
  \begin{itemize}
	\item Continue in what you have learned\pause
	\item knowing who taught you\pause
	\item known it from infancy he knew Scripture\pause
	\item they make one wise through \textgreek{p'istis} in the Anointed Jesus\pause
	\item All Scripture\ldots\pause
	\item so that the person of God may be equipped
  \end{itemize}
\end{frame}

\begin{frame}
  The stated uses are:\pause
  \begin{itemize}
	\item teaching \textgreek{didaskal'ian}\pause
	\item refutation \textgreek{>elegm'on}\pause
	\item correction \textgreek{>enan'orqwsin}\pause
	\item education \textgreek{paide'ian}
  \end{itemize}
\end{frame}

\begin{frame}
  Teaching: The act of imparting knowledge in a structured fashion.
\end{frame}

\begin{frame}
  Refutation: This is to demonstrate the error of an action or position either to the person or to others.
\end{frame}

\begin{frame}
  Correction: To bring someone over from error, to to make them straight.
\end{frame}

\begin{frame}
  Education: A catchall for training children and shaping their mind.
  It includes teaching, inculcating habits, even corrective punishment.
\end{frame}

\begin{frame}
  Taken together, this passage does not give several properties to Scripture we often assume it does.\pause
  \begin{itemize}
	\item Infallibility\pause
	  \begin{itemize}
		\item John 10.35\pause
	  \end{itemize}
	\item Foundation for truth\pause
	\item Sole source of divine revelation
  \end{itemize}
\end{frame}

\begin{frame}
  The Bible is, and must, have its inspiration in the original context and in community transmitting it.\pause
  \begin{itemize}
	\item Inspiration is \emph{communal}.\pause
	\item It draws its authority from whom we receive it.\pause
	\item It is intended for use within that community, ``the Church's book.''
  \end{itemize}
\end{frame}

\begin{frame}
  These are the characteristics we must assume when we consider its inspiration.
  It has more, but we know the ``more'' from tradition.
\end{frame}

\section{Word of God}

\subsection{John}

\begin{frame}
  The ``Word of God'' is an important title in the biblical text.
  It does \emph{not} belong to the Bible in the Bible.\pause
  \begin{itemize}
	\item ``Word of God'' is applied to a person.\pause
	\item ``Word of God'' is applied to the Gospel.\pause
	\item ``Words of God'' may be used for a message whether written or verbal.
  \end{itemize}
\end{frame}

\begin{frame}
  The best starting place for the ``Word of God'' is John:
  \begin{quote}
	In the beginning was the word, and the word was with God, and the Word was what God was.
	This one was in the beginning with God.
	Everything through him came to be, and without him came to be nothing that has come to be.
	In him was life, and the life was the light of humans,
	and the light shines in the darkness, and the darkness did not comprehend it.
  \end{quote}
\end{frame}

\begin{frame}
  Three things need to be noted here:\pause
  \begin{itemize}
	\item The Word is preexistent.\pause
	\item The Word is what God is but not identical.\pause
	\item The Word is a sort of principle that undergirds the world.
  \end{itemize}
\end{frame}

\begin{frame}
  John 10.35:\\ \pause
  \begin{quote}
	Jesus answered them, Is it not written in you all's law, I said you all are gods
	If God said of those, to whom the Word of God came, and the Scripture cannot be broken,
	why do you all say of him the Father hallowed and sent into the world, You blaspheme, because I said I am the Son of God?
  \end{quote}
\end{frame}

\begin{frame}
  The exchange goes:\pause
  \begin{itemize}
	\item Jesus declares he and the Father are one.\pause
	\item The Jews take up stones to kill him.\pause
	\item Jesus asks what he is being stoned for\pause
	\item They say that he is making himself God.\pause
	\item Our exchange takes place.
  \end{itemize}
\end{frame}

\begin{frame}
  This occurs previously in John 5.18.
  The biblical reference is to Psalm 81/82 which opens with ``God stands in the midst of the gods'' and reaches its climax with ``I said you all are gods\ldots ''
\end{frame}

\begin{frame}
  The original context for the Psalm was God judging the gods of the nations for failing to do their job right.
\end{frame}

\begin{frame}
  This cannot be the correct context for Jesus' statement.\pause
  \begin{itemize}
	\item The divine council is not a prominent motif in John.\pause
	\item Jesus uses the phrase ``Those to whom the word came,'' but neither it nor analogous phrases are ever used of angels.
  \end{itemize}
\end{frame}

\begin{frame}
  Ps.Jt. translates it as ``the assembly of the righteous.''
  John \emph{does} have a motif that Jesus is superior to the heroes of the OT.
  Ps.Jt. is later, but provides the correct context.
\end{frame}

\begin{frame}
  Jesus' argument boils down to:\pause
  \begin{itemize}
	\item The Word of God ``came'' to the prophets.\pause
	\item The Word made them divine.\pause
	\item If the Word made them divine, then the word itself must be divine, so Jesus is not out of order to call himself the Son of God.
  \end{itemize}
\end{frame}

\begin{frame}
  The Gospel of John frames the word as a person that is capable of divinizing others.
\end{frame}

\subsection{OT Precedent}

\begin{frame}
  There is OT precedent for the Word of God being a person.
\end{frame}

\begin{frame}
  After these things, the Word of the \textsc{Lord} appeared to Abram \emph{in a vision} saying, Do not fear Abram; I am shielding you. Your payment will be very great. Gen 15.1
\end{frame}

\begin{frame}
  Abraham you all's father rejoiced to see my day. He both saw and rejoiced.\\
  Then the Jews said to him, You aren't fifty years old, and you saw Abraham?\\
  Jesus said to them, Amen, amen I tell you all, before Abraham was, I am.
  Then they took up stones in order to throw them at him, but Jesus hid and went out of the Temple complex.
\end{frame}

\begin{frame}
  The Jews knew that Jesus was referring to Gen 15.
  The ``Word of God'' was used as an immanent \textsc{Yhwh} and thus a second \textsc{Yhwh}.
\end{frame}

\begin{frame}
  Two examples are that it was the Word of the Lord walking in the Garden (Gen 3.8) and that spoke to Moses from the burning bush (Ex 3.8).
\end{frame}

\begin{frame}
  For more information see \emph{The Jewish Targums and John's Logos Theology} by John Ronning
\end{frame}

\begin{frame}
  Another instance would be I Sam 3, where the Word of the Lord was scarce, because visions were scarce.
\end{frame}

\subsection{Gospel}

\begin{frame}
  The phrase ``Word of God'' is also used in the NT to denote the preaching of the Gospel.
\end{frame}

\begin{frame}
  Parable of the sower (Lk. 8.11) ``The seed is the Word of God.''\\
  Acts 4.31, ``They talked about the word of God with boldness.''\\
  ``Or did the Word of God come from you all, or did it reach you all only?'' I Cor 14.36
\end{frame}

\begin{frame}
  Thus when we get to the ``sword of the Spirit,'' the ``Word of God'' is the proclamation of the Gospel (Eph. 6.17)
\end{frame}

\begin{frame}
  It does sometimes refer more generically to a message or a command (cf. Pr. 30.5 and Mk 7.13).
\end{frame}

\begin{frame}
  Significance of using it for ``Gospel'' lies in the meaning of ``Gospel.''
\end{frame}

\begin{frame}
  The Gospel is \textbf{not}\pause
  \begin{itemize}
	\item Not having to work for salvation\pause
	\item That Jesus is offering you a ``personal'' relationship\pause
	\item That we may go to the throne with boldness.\pause
  \end{itemize}
  None of these are about Jesus.
  They are about us, but the Gospel must be Christ-centered.
\end{frame}

\begin{frame}
  The Gospel is:
  \begin{quote}
	Who [Christ], being in the form of God did not reckon equality with God something to be clung to,
	but emptied himself taking the form of a slave, being into the likeness of humans,
	humbled himself becoming obedient to the point of death, even a death by a cross.
	Therefore God has highly exalted him, and he gave him the name that is above every name,
	so that at the name of Jesus every knee should bend those that are heavenly, and earthly, and subterranean,
	and  every tongue enthusiastically agree that Jesus the Anointed is Lord, to the glory of Father God. -- Phil 2.5-11
  \end{quote}
\end{frame}

\begin{frame}
  Jesus is ``light of light, true God of true God\ldots came down from heaven and became man\ldots was crucified for us under Pontius Pilate and suffered and buried, and he rose on the third day according to the Scriptures.
  He ascended into heaven and is seated at the right hand of the Father, and he will come again to judge the living and the dead.''
\end{frame}

\begin{frame}
  Getting from there to the message of the Gospel being the Word of God requires us to work backwards.
\end{frame}

\begin{frame}
  St. Paul argues for the unity of the Body saying, ``you are the Body of Christ'' in I Cor. 12.
  His intention is to argue this by making the point:\pause
  \begin{itemize}
	\item He states divisions are a prime concern (1.10)\pause
	\item Individual members are incomplete (12.13)\pause
	\item Immorality defiles their membership, because what they do with their physical bodies they do to Christ's body (I Cor 6.15)\pause
  \end{itemize}
\end{frame}

\begin{frame}
  To justify this unity he proposes two methods by which they become one body:\pause
  \begin{itemize}
	\item They are baptized into one body by one Spirit (12.13).\pause
	\item They share the same Eucharist, ``because there is one bread, we, who are many, are one body, for we all are partakers of the one bread.'' (I Cor 10.17)
  \end{itemize}
\end{frame}

\begin{frame}
  Similarly, Luke has some disciples going to Emmaus not recognize Jesus until they break bread together (Lk 24.13ff.)
  They weren't able to recognize him (v. 16) until they broke bread together (30-1).
\end{frame}

\begin{frame}
  This is such that:
  \begin{quote}
	Doubtless, great is the mystery of religion,\\
	God was revealed in flesh, vindicated in the Spirit, seen by angels, preached to the nations, trusted in the world, received up in glory. I Tim 3.16
  \end{quote}
\end{frame}

\begin{frame}
  The proclamation of the Gospel is the proclamation of Christ's kingship.
  Baptism is a proclamation of loyalty to Christ the King.
  The Eucharist, alongside incense, are the remaining liturgical offerings we bring to God.
  They manifest the rule of God and actualize it so that rich and poor, men and women, join together as one body.
  They then corporately experience the Word of God.
\end{frame}

\begin{frame}
  This brings us back to Paul's words to Timothy.
  He has known the words of God since his infancy, and he has known them in this context.
  Thus he knows them within the corporate experience and knowledge of God.
\end{frame}

\section{Conclusion}

\begin{frame}
  Inspiration goes:\pause
  \begin{itemize}
	\item The Word of God properly is a person.\pause
	\item The revelation is the experience of God given to the man of God.\pause
	\item The Bible reveals those experiences and/or their results to us.\pause
	\item The Word is made present corporately at the Lord's Supper.\pause
	\item The Bible's inspiration goes from its authors, to its transmitters, to the believing community.
  \end{itemize}
\end{frame}

\end{document}
